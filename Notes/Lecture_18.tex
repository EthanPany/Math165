\section{Lecture 18 \& 19: Kernel, Range, Eigenvalues}
\rmk{
    This lecture covers: 
    \begin{itemize}
        \item 6.1 Definition of Linear Transformations
        \item 6.2 Transformations of $\mathbb{R}^2$
        \item 6.3 The Kernel and Range of a Linear Transformation
    \end{itemize}
}

\subsection{Definition of Linear Transformations}
\defn{Mapping}{
    Let $V$ and $W$ be vector spaces. A \textbf{mapping} $T$ from $V$ to $W$ is a rule that assigns to each vector $\vec{v}$ in $V$ precisely one vector $\vec{w} = T(\vec{v})$. We write $T: V \to W$.
}

\defn{Linear Transformation}{
    Let $V$ and $W$ be vector spaces over the same field. A mapping $T: V \to W$ is a \textbf{linear transformation} if for all $\vec{v}_1, \vec{v}_2 \in V$ and all scalars $c$:
    \begin{enumerate}
        \item $T(\vec{u} + \vec{v}) = T(\vec{u}) + T(\vec{v})$ for all $\vec{u}, \vec{v} \in V$
        \item $T(c\vec{v}) = cT(\vec{v})$ for all $\vec{v} \in V$
    \end{enumerate}
    In the above equations, the operations on the left of the equal signs are the ones defined in the domain $V$ and
the ones on the right of the equal signs are the ones defined in the codomain $W$.
}

\thmr{}{}{
    Let $V, W$ be vector spaces over field $F$. A mapping $T: V \to W$ is a linear transformation if and only if for all $\lambda_1, \lambda_2 \in F$ and all $\vec{v}_1, \vec{v}_2 \in V$:
    \[
        T(\lambda_1\vec{v}_1 + \lambda_2\vec{v}_2) = \lambda_1T(\vec{v}_1) + \lambda_2T(\vec{v}_2)
    \]
}

\ex{
    Show $T:P_2 \to P_4$ given by $T(p) = x^2 p(x)$ is linear. 
}







\newpage