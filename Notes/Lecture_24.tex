\section{Lecture 24: First Order Linear Systems}
\rmk{
    This lecture covers: 
    \begin{itemize}
        \item 9.1 First Order Linear Systems
        \item 9.2 Vector Formulation
        \item 9.3 General Results for First Order Linear Systems
    \end{itemize}
}

\rmkb{
    Today we are going to cover 9.1 to 9.3. We are going to skip many theorems. \\
    Next week we are going to do some examples for chapter 8. 
}

\subsection{Chapter 9.1: First Order Linear Systems}


\[
    \begin{bmatrix}
        x_1'(t) = a_{11}(t)x_1(t) + a_{12}(t)x_2(t) + \cdots + a_{1n}(t)x_n(t) + b_1(t) \\
        x_2'(t) = a_{21}(t)x_1(t) + a_{22}(t)x_2(t) + \cdots + a_{2n}(t)x_n(t) + b_2(t) \\
        \cdots \\
        x_n'(t) = a_{n1}(t)x_1(t) + a_{n2}(t)x_2(t) + \cdots + a_{nn}(t)x_n(t) + b_n(t) \\
    \end{bmatrix}
\]
The $b_i$ is the consists of the non-homogenous part.
If $b_i (t) = 0$, the system is homogenous. 



An example of the first-order linear system is:

\ex{
    \[
        x_1 ' = x_1 + 2x_2
    \]
    \[
        x_2 ' = 2x_1 + 2x_2
    \]
    \[
        x_1(0) = 1, x_2(0) = 0
    \]
}
The \textbf{Initial value} here is $x_1(0) = 1, x_2(0) = 0$.

A \textbf{solution} is an ordered n-tuple of functions $x_1(t), x_2(t), \dots, x_n(t)$ that satisfies the system of equations.

The solution will be in the form of: 
\ex{
    % \[
    %     x_1 ' = x_1 + 2x_2
    % \]
    % \[
    %     x_2 ' = 2x_1 + 2x_2
    % \]
    % \[
    %     x_1(0) = 1, x_2(0) = 0
    % \]
    \[
        x_1(t) = \text{Some function}, x_2(t) = \text{Some function}
    \]
}
Which is a vector of functions. 

There's a clever trick to solving the 2 by 2 system using the derivative, however, it is not that fast and no one use it in the exam so we are going to skip it. 


\rmkb{
    The first-order linear system is restrictive. However, we can transform a higher-order system by renaming functions.
}

\ex{
    \[
        \frac{d^2x}{dt^2} + 4e^t \frac{dx}{dt} - 9t^2x = 7t^2
    \]
    Strataegy: Let $x_1 = x$, $x_1' = x_2$
    \[
        x_1' = x_2
    \]
    \[
        x_2' = 9t^2x + 4e^t x_2 - 7t^2
    \]
    (Chapter 9.1)
}


\subsection{Chapter 9.2: Vector Formulation}
\rmkb{Chapter 9.2: How to transform the system into a matrix form.
}

\ex{
    Formulate with vectors. \\
    \[
        \vec{x}(t) = \begin{bmatrix}
            x_1(t) \\
            x_2(t) \\
            \cdots \\
            x_n(t)
        \end{bmatrix}
    \]
    \[
        \vec{x}'(t) = \begin{bmatrix}
            x_1'(t) \\
            x_2'(t) \\
            \cdots \\
            x_n'(t)
        \end{bmatrix}
    \]
    % \[
    %     A(t) = \begin{bmatrix}
    %         a_{11}(t) & a_{12}(t) & \cdots & a_{1n}(t) \\
    %         a_{21}(t) & a_{22}(t) & \cdots & a_{2n}(t) \\
    %         \cdots & \cdots & \cdots & \cdots \\
    %         a_{n1}(t) & a_{n2}(t) & \cdots & a_{nn}(t) 
    %     \end{bmatrix}

    % \]
    \[
        A(t) = \begin{bmatrix}
            a_{11}(t) & a_{12}(t) & \cdots & a_{1n}(t) \\
            a_{21}(t) & a_{22}(t) & \cdots & a_{2n}(t) \\
            \cdots & \cdots & \cdots & \cdots \\
            a_{n1}(t) & a_{n2}(t) & \cdots & a_{nn}(t)
        \end{bmatrix}
    \]
    Coefficient matrix: \\
    \[
        \vec{b}(t) = \begin{bmatrix}
            b_1(t) \\
            b_2(t) \\
            \cdots \\
            b_n(t)
        \end{bmatrix}
    \]

}   

\thmr{}{}{
    $V_n(I)$ is a column vector of $n$ functions defined on an interval $I$.
}

\newpage

\ex{
    \[
        \begin{bmatrix}
            e^{3t} \\
            2\\
            e^{7t}
        \end{bmatrix}
        \in V_3(\mathbb{R})
    \]
    for any fixed $n$, $I$, $V_n(I)$ is a vector space.
}




\subsubsection{Wronskian}
\rmkb{
    We do not test this
}
Wronskan of a set of $n$ column vectors in $V_n(I)$ 

Wronskien of $\{\vec{x}_1, \vec{x}_2, \cdots, \vec{x}_n \}$

$W(t) = \text{Wronskian} = \begin{bmatrix}
    \vec{x}_1 & \vec{x}_2 & \cdots & \vec{x}_n
\end{bmatrix}  $

\thmr{}{}{
    If $W(t) \neq 0$ for all $t \in I$, then $\{\vec{x}_1, \vec{x}_2, \cdots, \vec{x}_n \}$ is linearly independent. 
}

\thmr{}{}{
    If $\{\vec{x}_1, \vec{x}_2, \cdots, \vec{x}_n \}$ is linearly independent, then $W(t) \neq 0$ for all $t \in I$. 
}


\subsection{Chapter 9.3: Genearl results for first-order linear systems}
\thmr{Initial value problem}{}{
    $\vec{x}'(t) = A(t) \vec{x}(t) + \vec{b}(t)   $ and $\vec{x}(t_0) = \vec{x}_0$ has a unique solution on an interval $I$ containing $t_0$ if $A(t)$ and $\vec{b}(t)$ are continuous on $I$.
}   



\thmr{9.3.2}{}{
    The set of solutions to the homogenous system $\vec{x}'(t) = A(t) \vec{x}(t)$ is a vector space of dimension $n$.
}

\subsubsection{Fundamental solution set}
The fundamental solution set is basically a basis for the solution space.

\thmr{The fundamental solution set}{}{
    \[
        S = \{
            x_1, x_2, \cdots, x_n
        \}
    \]
    S is a set of solutions to the homogenous system $\vec{x}'(t) = A(t) \vec{x}(t)$ that are linearly independent.
}

The non-homogenous case:
\[
    \vec{x} = c_1 \vec{x}_1 + c_2 \vec{x}_2 + \cdots + c_n \vec{x}_n + \boxed{\vec{x}_p}
\]

We add a single solution to the homogenous system to the solution of the non-homogenous system.


Simplifying assumptions:
\[
    \vec{x}'(t) = A\vec{x}(t)
\]
is homogenous and $A$ is a matrix of constants and $A$ is non-defective.

\rmkb{
    Recall: Non-defective means that the matrix has $n$ linearly independent eigenvectors.
}

\thmr{}{}{
    Let $A$ be a $n \times n$ matrix of real constants, and let $\lambda$ be a real eigenvalue of corresponding to the eigenvector $\vec{v}$. \\
    
    Then 
    \[
        \vec{x}(t) = e^{\lambda t} \vec{v}
    \]
    is a solution to the homogenous system $\vec{x}'(t) = A\vec{x}(t)$.
}
\rmkb{
    Recall: Means that $A\vec{v} = \lambda \vec{v}$
}

\pf{
    If $\vec{x}(t) = e^{\lambda t} \vec{v}$, then
    \[
        A(e^{\lambda t} \vec{v}) = e^{\lambda t} A\vec{v} = e^{\lambda t} \lambda \vec{v} = \lambda e^{\lambda t} \vec{v}
    \]
    On the other side, if I take the derivative of $\vec{x}(t)$, I get
    \[
        \vec{x}'(t) = \lambda e^{\lambda t} \vec{v}
    \]
    We notice that the two sides are equal.
}

And we are going to have enough solutions to form a basis for the solution space.


\thmr{}{}{
    If $A$ has $n$ real eigenvectors $\vec{v}_1, \vec{v}_2, \cdots, \vec{v}_n$ with corresponding real eigenvalues $\lambda_1, \lambda_2, \cdots, \lambda_n$, then 
    \[
        \vec{x}_k = e^{\lambda_k t} \vec{v}_k
    \]
    is a solution to the homogenous system $\vec{x}'(t) = A\vec{x}(t)$.

}
\newpage

\ex{
    \[
        A = \begin{bmatrix}
            1 & 2 \\
            2 & -2
        \end{bmatrix}
    \]
    The eigenvalue is: \\
    \[
        | A - \lambda I | = (1 - \lambda) \cdot (-2 - \lambda) - 4 = 0
    \]
    \[
        \lambda = 2, -3
    \]
    nullspace of $A - 2I$ is $\begin{bmatrix}
        -1 & 2 \\
        2 & -4
    \end{bmatrix} \to \begin{bmatrix}
        1 & -2 \\
        0 & 0
    \end{bmatrix} 
    $ \[
        \vec{v}_1 = \begin{bmatrix}
            2 \\
            1
        \end{bmatrix}
    \]
    nullspace of $A + 3I$ is $\begin{bmatrix}
        4 & 2 \\
        2 & 1
    \end{bmatrix} \to \begin{bmatrix}
        2 & 1 \\
        0 & 0
    \end{bmatrix}
    $ \[
        \vec{v}_2 = \begin{bmatrix}
            -1 \\
            2
        \end{bmatrix}
    \]
    \textbf{What's new today:} \\
    Solution = \\
    \[
        \begin{bmatrix}
            x_1 \\ x_2
        \end{bmatrix}
        =
        c_1 e^{2t} \begin{bmatrix}
            2 \\ 1
        \end{bmatrix}
        +
        c_2 e^{-3t} \begin{bmatrix}
            -1 \\ 2
        \end{bmatrix}
    \]
    \[
        x_1 = 2 c_1 e^{2t} - c_2 e^{-3t}
    \]
    \[
        x_2 = c_1 e^{2t} + 2c_2 e^{-3t}
    \]

    Initial values: \\
    \[
        x_1(0) = 2c_1 - c_2 = 1
    \]
    \[
        x_2(0) = c_1 + 2c_2 = 0
    \]
    \[
        \begin{bmatrix}
            \begin{array}{cc|c}
                2 & -1 & 1 \\
                1 & 2 & 0
            \end{array}
        \end{bmatrix}
        \to
        \begin{bmatrix}
            \begin{array}{cc|c}
                1 & 2 & 0 \\
                2 & -1 & 1
            \end{array}
        \end{bmatrix}
        \to
        \begin{bmatrix}
            \begin{array}{cc|c}
                1 & 2 & 0 \\
                0 & -5 & 1
            \end{array}
        \end{bmatrix}
        \to
        \begin{bmatrix}
            \begin{array}{cc|c}
                1 & 0 & 2/5 \\
                0 & 1 & -1/5
            \end{array}
        \end{bmatrix}
    \]

}


\ex{
    \[
        \vec{x' = A\vec{x}}
    \]
    \[
        A = \begin{bmatrix}
            0 & 4 \\
            1 & 0
        \end{bmatrix}
    \]
    \[
        x_1' = 4x_2
    \]
    \[
        x_2' = x_1
    \]
    \textbf{Step 1: Find the eigenvalues and eigenvectors} \\
    \[
        |A - \lambda I| = \begin{bmatrix}
            -\lambda & 4 \\
            1 & -\lambda
        \end{bmatrix} = \lambda^2 - 4 = 0
    \]
    \[
        \lambda _1 = 2, \lambda_2 = -2
    \]
    \textbf{For $\lambda_1 = 2$} \\
    \[
        \begin{bmatrix}
            -2 & 4 \\
            1 & -2
        \end{bmatrix} \to \begin{bmatrix}
            1 & -2 \\
            0 & 0
        \end{bmatrix}
    \]
    \[
        \boxed{\vec{v}_1 = \begin{bmatrix}
            2 \\
            1
        \end{bmatrix}}
    \]
    \textbf{For $\lambda_2 = -2$} \\
    \[
        \begin{bmatrix}
            2 & 4 \\
            1 & 2
        \end{bmatrix} \to \begin{bmatrix}
            1 & 2 \\
            0 & 0
        \end{bmatrix}
    \]
    \[
        \boxed{\vec{v}_2 = \begin{bmatrix}
            -2 \\
            1
        \end{bmatrix}}
    \]
    \textbf{Step 2: Write the general solution} \\
    \[
        \begin{bmatrix}
            x_1 \\ x_2
        \end{bmatrix}
        =
        c_1 e^{2t} \begin{bmatrix}
            2 \\ 1
        \end{bmatrix}
        +
        c_2 e^{-2t} \begin{bmatrix}
            -2 \\ 1
        \end{bmatrix}
    \]
}

\rmkb{
    If you can not zero out the second role, it means that you are wrong. 
}


Let's do a 3 by 3 example.
We will leave the complex case for the next lecture.s

\ex{
    \[
        A = \begin{bmatrix}
            0 & -3 & 1 \\
            -2 & -1 & 1 \\
            0 & 0& 2
        \end{bmatrix}
    \]
    \textbf{Step 1: Find the eigenvalues and eigenvectors} \\
    \[
        |A - \lambda I| = 
        \begin{bmatrix}
            0 - \lambda & -3 & 1 \\
            -2 & -1 - \lambda & 1 \\
            0 & 0 & 2 - \lambda
        \end{bmatrix}
        = 0
    \]
    Characteristic polynomial: \\
    \[
        (2 - \lambda)(\lambda -3) (\lambda - 2) = 0
    \]
    \[
        \lambda_1 = 2, \lambda_2 = -3, \lambda_3 = 2
    \]
    \textbf{For $\lambda_1 = 2$} \\
    \[
        \begin{bmatrix}
            -2 & -3 & 1 \\
            -2 & -3 & 1 \\
            0 & 0 & 0
        \end{bmatrix} \to \begin{bmatrix}
            -2 & -3 & 1 \\
            0 & 0 & 0 \\
            0 & 0 & 0
        \end{bmatrix}
    \]
        $x_2, x_3 $ are free
    \[
        -2x_1 = 3x_2 - x_3
    \]
    \[
        x_1 = -\frac{3}{2} x_2 + \frac{1}{2} x_3
    \]

    \[
        \vec{v}_1 = (3, -2, 0)
    \]
    \[
        \vec{v}_2 = (1, 0, 2)
    \]

    \textbf{For $\lambda_2 = -3$} \\
    \[
        \begin{bmatrix}
            3 & -3 & 1 \\
            -2 & 2 & 1 \\
            0 & 0 & 5
        \end{bmatrix} \to \begin{bmatrix}
            1 & -1 & 0 \\
            0 & 0 & 1\\
            0 & 0 & 0
        \end{bmatrix}
    \]
    $x_2$ is free
    \[
        x_1 = x_2
    \]
    \[
        x_3 = 0
    \]
    \[
        \vec{v}_3 = (1, 1, 0)
    \]

    \textbf{Step 2: Write the general solution} \\
    \[
        \begin{bmatrix}
            x_1 \\ x_2 \\ x_3
        \end{bmatrix}
        =
        c_1 e^{2t} \begin{bmatrix}
            3 \\ -2 \\ 0
        \end{bmatrix}
        +
        c_2 e^{-3t} \begin{bmatrix}
            1 \\ 0 \\ 2
        \end{bmatrix}
        +
        c_3 e^{2t} \begin{bmatrix}
            1 \\ 1 \\ 0
        \end{bmatrix}
    \]

}









\newpage